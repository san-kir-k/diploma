% Также можно использовать \Referat, как в оригинале
\begin{abstract}

    Общий объём работы составляет \pageref{LastPage}\,страниц%
    \ifnum \totfig >0
    , \totfig~рисунков%
    \fi
    \ifnum \tottab >0
    , \tottab~табл.%
    \fi
    %
    \ifnum \totbib >0
    , \totbib~источников%
    \fi
    %
    \ifnum \totapp >0
    , \totapp~прил.%
    \else
    .%
    \fi

    ~
    
    МАТРИЦА АДАМАРА, АЛГОРИТМ ПОИСКА МАТРИЦ АДАМАРА С ВОЗВРАТОМ, ЭКВИВАЛЕНТНОСТЬ ПО ХОЛЛУ, АЛГОРИТМ ПОИСКА МИНИМАЛЬНОЙ МАТРИЦЫ АДМАРА, Q-ЭКВИВАЛЕНТНОСТЬ МАТРИЦ, АЛГОРИТМ ПОИСКА ПРЕДСТАВИТЕЛЕЙ КЛАССОВ Q-ЭКВИВАЛЕНТНОСТИ, ТЕОРИЯ КОДИРОВАНИЯ.
    
    ~
    
    Объектом исследования в данной работе являются матрицы Адамара до 64-го порядка, использующиеся в кодировании информации обработке дискретных сигналов. Предметом исследования являются методы их построения и классификации.
    
    В данной работе реализованы алгоритм построения матриц Адамара с возвратом, алгоритм поиска минимальной матрицы Адамара и алгоритм поиска представителей классов Q-эквивалентности матриц Адамара.
    
    В теоритической части работы дается описание основных математических объектов, операций, теорем и подходов, используемых в вышеуказанных алгоритмах.
    
    В практической части приведена реализация алгоритмов, рассмотрены некоторые способы оптимизации программной реализации для уменьшения времени работы, производится сравнение результатов работы вариаций алгоритмов.
    
    Основным результатом данной работы является эффективная реализация алгоритма поиска представителей классов Q-эквивалентности.
\end{abstract}
