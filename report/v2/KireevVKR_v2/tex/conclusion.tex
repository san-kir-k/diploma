\Conclusion

В данной выпускной квалификационной работе были полностью выполнены все поставленные задачи для матриц до $28$-го порядка включительно: был реализован алгоритм построения матриц Адамара малых порядков, были реализованы и оптимизированы алгоритм приведения матрицы Адамара к минимальному виду и алгоритм поиска представителей классов Q-эквивалентности матриц Адамара, также было проведено сравнение производительности различных реализаций алгоритмов.

Предложенная реализация алгоритма поиска представителей классов Q-эквивалентности матриц Адамара может быть полезна для более эффективного получения неэквивалентных по Холлу матриц, а также их классификации по конструктивным особенностям.

Возможным развитием данной работы является продолжение алгритма на более высокие порядки. Для этого придется задействовать большее число процессоров или даже объединить набор компьютеров в общий вычислительный ресурс, не используя механизмы синхронизации через разделяемую память.
