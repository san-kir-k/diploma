\Introduction

Данная выпускная квалификационная работа рассматривает матрицы Адамара, способы их построения и классификации. Данные матрицы имеют большое значение в алгебре, комбинаторике, криптографии, теории кодирования, также играют важную роль в теории дискретных сигналов. 

С момента появления матриц Адамара как математических объектов прошло много времени, люди находили все более новые практические применения матрицам Адамара, и, соответственно, росла потребность в выявлении новых матриц, которые могли бы иметь интересные свойства. Однако задача построения новых матриц Адамара в общем случае является вычислительно сложной, и даже с современными компьютерными мощностями люди смогли полностью обработать лишь те матрицы, чей порядок не превышает 32.

Целью данной работы является реализация алгоритма поиска представителей классов Q-эквивалентности матриц Адамара малых порядков. Данный алгоритм позволит находить всех представителей данного Q-класса и строить неэквивалентные матрицы Адамара. Предполагается, что алгоритм будет эффективным и позволит получить новые результаты о количестве и структуре матриц Адамара малых порядков.

Актуальность данной работы обусловлена тем, что эквивалентность матриц Адамара имеет важное значение для практического использования таких матриц в различных областях. Например, матрицы Адамара используются для построения кодов Голея, которые являются оптимальными кодами с исправлением ошибок. Для эффективного кодирования и декодирования информации необходимо выбирать матрицы Адамара из одного класса эквивалентности, чтобы избежать повторений и неоднозначностей.

Кроме того, матрицы Адамара используются для построения систем ортогональных функций, которые применяются в обработке сигналов и спектральном анализе. Для получения различных систем ортогональных функций необходимо изучать разные классы эквивалентности матриц Адамара.
