%% Преамбула TeX-файла

% 1. Стиль и язык
\documentclass[times, 14pt]{G7-32} % Стиль (по умолчанию размер шрифта 14pt)

% Остальные стандартные настройки убраны в preamble.inc.tex.
\include{preamble.inc}

% =======================================================
% Математика:
\usepackage{mathtools, cancel, physics, euscript, xfrac, amsmath, amsthm}

\newtheorem{Th}{Теорема}
\newtheorem{Lm}{Лемма}
\newtheorem{Df}{Определение}
\newtheorem{Ex}{Пример}
\newtheorem{Rm}{Ремарка}
\newtheorem{Alg}{Алгоритм}
\newtheorem{Ax}{Аксиома}
\newtheorem{Crl}{Следствие}
\newtheorem{Prp}{Гипотеза}
% =======================================================
\usepackage{subcaption}
% =======================================================
% Псевдокод:
\usepackage{algorithm}
\usepackage{algpseudocode}
\floatname{algorithm}{Алгоритм}
\algrenewcommand\algorithmicwhile{\textbf{До тех пока}}
\algrenewcommand\algorithmicdo{\textbf{выполнять}}
\algrenewcommand\algorithmicrepeat{\textbf{Повторять}}
\algrenewcommand\algorithmicuntil{\textbf{Пока выполняется}}
\algrenewcommand\algorithmicend{\textbf{Конец}}
\algrenewcommand\algorithmicif{\textbf{Если}}
\algrenewcommand\algorithmicelse{\textbf{иначе}}
\algrenewcommand\algorithmicthen{\textbf{тогда}}
\algrenewcommand\algorithmicfor{\textbf{Цикл}}
\algrenewcommand\algorithmicforall{\textbf{Выполнить для всех}}
\algrenewcommand\algorithmicfunction{\textbf{Функция}}
\algrenewcommand\algorithmicprocedure{\textbf{Процедура}}
\algrenewcommand\algorithmicloop{\textbf{Зациклить}}
\algrenewcommand\algorithmicrequire{\textbf{Условия:}}
\algrenewcommand\algorithmicensure{\textbf{Обеспечивающие условия:}}
\algrenewcommand\algorithmicreturn{\textbf{Возвратить}}
\algrenewtext{EndWhile}{\textbf{Конец цикла}}
\algrenewtext{EndLoop}{\textbf{Конец зацикливания}}
\algrenewtext{EndFor}{\textbf{Конец цикла}}
\algrenewtext{EndFunction}{\textbf{Конец функции}}
\algrenewtext{EndProcedure}{\textbf{Конец процедуры}}
\algrenewtext{EndIf}{\textbf{Конец условия}}
\algrenewtext{EndFor}{\textbf{Конец цикла}}
\algrenewtext{BeginAlgorithm}{\textbf{Начало алгоритма}}
\algrenewtext{EndAlgorithm}{\textbf{Конец алгоритма}}
\algrenewtext{ElsIf}{\textbf{иначе если }}
% =======================================================
% Листинг кода:
\usepackage{verbatim}
\usepackage{listings}
\usepackage{listingsutf8}
\lstset{
    basicstyle=\small\ttfamily,
    numbers=left,
    frame=single,
    breaklines=true,
    breakatwhitespace=false,
    extendedchars=\true,
    keepspaces=true
}
\newcommand\lstlistinginline[1]{{\lstlistingstyle\lstinline!#1!}}
% =======================================================

\begin{document}

\frontmatter

% Также можно использовать \Referat, как в оригинале
\begin{abstract}

    Общий объём работы составляет \pageref{LastPage}\,страниц%
    \ifnum \totfig >0
    , \totfig~рисунков%
    \fi
    \ifnum \tottab >0
    , \tottab~табл.%
    \fi
    %
    \ifnum \totbib >0
    , \totbib~источников%
    \fi
    %
    \ifnum \totapp >0
    , \totapp~прил.%
    \else
    .%
    \fi

    ~
    
    МАТРИЦА АДАМАРА, АЛГОРИТМ ПОИСКА МАТРИЦ АДАМАРА С ВОЗВРАТОМ, ЭКВИВАЛЕНТНОСТЬ ПО ХОЛЛУ, АЛГОРИТМ ПОИСКА МИНИМАЛЬНОЙ МАТРИЦЫ АДМАРА, Q-ЭКВИВАЛЕНТНОСТЬ МАТРИЦ, АЛГОРИТМ ПОИСКА ПРЕДСТАВИТЕЛЕЙ КЛАССОВ Q-ЭКВИВАЛЕНТНОСТИ, ТЕОРИЯ КОДИРОВАНИЯ.
    
    ~
    
    Объектом исследования в данной работе являются матрицы Адамара до 64-го порядка, использующиеся в кодировании информации обработке дискретных сигналов. Предметом исследования являются методы их построения и классификации.
    
    В данной работе реализованы алгоритм построения матриц Адамара с возвратом, алгоритм поиска минимальной матрицы Адамара и алгоритм поиска представителей классов Q-эквивалентности матриц Адамара.
    
    В теоритической части работы дается описание основных математических объектов, операций, теорем и подходов, используемых в вышеуказанных алгоритмах.
    
    В практической части приведена реализация алгоритмов, рассмотрены некоторые способы оптимизации программной реализации для уменьшения времени работы, производится сравнение результатов работы вариаций алгоритмов.
    
    Основным результатом данной работы является эффективная реализация алгоритма поиска представителей классов Q-эквивалентности.
\end{abstract}


\tableofcontents

% Автоматический список сокращений:
% \printnomenclature

% Введение:
\Introduction

Данная выпускная квалификационная работа рассматривает матрицы Адамара, способы их построения и классификации. Данные матрицы имеют большое значение в алгебре, комбинаторике, криптографии, теории кодирования, также играют важную роль в теории дискретных сигналов. 

С момента появления матриц Адамара как математических объектов прошло много времени, люди находили все более новые практические применения матрицам Адамара, и, соответственно, росла потребность в выявлении новых матриц, которые могли бы иметь интересные свойства. Однако задача построения новых матриц Адамара в общем случае является вычислительно сложной, и даже с современными компьютерными мощностями люди смогли полностью обработать лишь те матрицы, чей порядок не превышает 32.

Целью данной работы является реализация алгоритма поиска представителей классов Q-эквивалентности матриц Адамара малых порядков. Данный алгоритм позволит находить всех представителей данного Q-класса и строить неэквивалентные матрицы Адамара. Предполагается, что алгоритм будет эффективным и позволит получить новые результаты о количестве и структуре матриц Адамара малых порядков.

Актуальность данной работы обусловлена тем, что эквивалентность матриц Адамара имеет важное значение для практического использования таких матриц в различных областях. Например, матрицы Адамара используются для построения кодов Голея, которые являются оптимальными кодами с исправлением ошибок. Для эффективного кодирования и декодирования информации необходимо выбирать матрицы Адамара из одного класса эквивалентности, чтобы избежать повторений и неоднозначностей.

Кроме того, матрицы Адамара используются для построения систем ортогональных функций, которые применяются в обработке сигналов и спектральном анализе. Для получения различных систем ортогональных функций необходимо изучать разные классы эквивалентности матриц Адамара.


\mainmatter

% Основной текст:
\include{text}

\backmatter

% Заключение:
\Conclusion

В данной выпускной квалификационной работе были полностью выполнены все поставленные задачи для матриц до $28$-го порядка включительно: был реализован алгоритм построения матриц Адамара малых порядков, были реализованы и оптимизированы алгоритм приведения матрицы Адамара к минимальному виду и алгоритм поиска представителей классов Q-эквивалентности матриц Адамара, также было проведено сравнение производительности различных реализаций алгоритмов.

Предложенная реализация алгоритма поиска представителей классов Q-эквивалентности матриц Адамара может быть полезна для более эффективного получения неэквивалентных по Холлу матриц, а также их классификации по конструктивным особенностям.

Возможным развитием данной работы является продолжение алгритма на более высокие порядки. Для этого придется задействовать большее число процессоров или даже объединить набор компьютеров в общий вычислительный ресурс, не используя механизмы синхронизации через разделяемую память.


% Список источников:
\bibliographystyle{ugost2008}
\bibliography{res/references}

% Приложения:
% \appendix

% \chapter{Картинки}
\label{cha:appendix1}

\blindtext

\begin{figure}
    \centering
    \includegraphics[scale=3]{example-grid-100x100pt}
    \caption{Картинка в приложении}
\end{figure}


% \chapter{Еще картинки}
\label{cha:appendix2}

\blindtext

\begin{figure}
    \centering
    \includegraphics{example-image-golden}
    \caption{Картинка в приложении}
\end{figure}


\end{document}